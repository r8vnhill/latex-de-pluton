%% "Makarena" (c) by Ignacio Slater M.
%% "Makarena" is licensed under a
%% Creative Commons Attribution 4.0 International License.
%% You should have received a copy of the license along with this
%% work. If not, see <https://creativecommons.org/licenses/by/4.0/>.
\documentclass{article}
  \usepackage{import} % This will make our life easier
  \import{preamble}{Packages}
  \import{preamble}{Definitions}
  \import{preamble}{config}

  \setup { 
    title = {
      A Heuristic Proof on the Non-Deterministic Behaviour of the \textit{Little Room}'s Printer
    },
    subtitle = { Thesis Proposal for the degree of Ph.D in Printer Sciences },
    author = {
      \textbf{DankMasterBlader} \\
      \textit{Institute of Printer Sciences} \\
      Letter Paper University of Applied Sciences
    },
    advisors = {
      \textbf{Alf} \\
      \textit{?¿??} \\
      University of \texttt{[BLANK]}. Ph.D  \\
      {\small\textit{Thesis Advisor}},
      \textbf{Juan Carlos Bodoque} \\
      \textit{Thirty-One Minutes} \\
      {\small\textit{Reviewer}}
    },
    logo = {logos/Logo_de_Plut_n.png},
    location = {Brooklyn, Japan},
    date = \today,
    % Please, I ask you for all that's precious in the world, always include a version for your 
    % document.
    version = 0,
    build = 1,
  }

\begin{document}
  \begin{titlepage}
    \centering
    % Some vertical space before the title (note the *, it's needed to add space at the beginning 
    % of the page).
    \vspace*{2cm}
    \titleblock [2cm]
    \inputlogo [3.5cm]
    \vspace{1cm}  % Some vertical space after the logo.
    \authorblock
    \vfill  % Fills the page so the location and date are at the bottom.
    \location \\
    \dateblock \\
    \footnotesize { \texttt{\fullversion} }
  \end{titlepage}

  \import{contents}{Abstract.tex}
  \newpage
  \tableofcontents
  \newpage
  \section{Introduction (or Why I Created this Template)}
    This template was born because there's one big problem with \LaTeX: it's very difficult to start
    learning it.
    But the thing is, it's not because \LaTeX\ is overcomplicated, it's because there's little to no
    material to learn.

    This is an ambitious task, because I want you to be able to use this template as-is with minimal
    \LaTeX\ knowledge, but also give you the tools to learn it and to understand how this template is
    made.
    It doesn't matter if you're a beginner or an advanced \LaTeX\ user, this template is aimed at 
    everyone.
\end{document}
