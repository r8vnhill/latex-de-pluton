\section{ Quickstart }
  For the ones that want to just use this template and don't care about the details, this is the 
  section you'd want to focus on.

  \subsection{ Using the template with Overleaf }
    \begin{itemize}
      \item \textbf{Step 1:} \textit{Download the repository} as a zip file. 
        Alternatively, you can clone the repository and zip it manually.
      \item \textbf{Step 2:} \textit{In Overleaf, create a new project} and \textit{import} the zip 
        file.
        \texttt{New Project > Upload project}.
      \item \textbf{Step 3:} In the \textit{Project Menu} go to \texttt{Settings > Compiler} and 
        select \texttt{XeLaTeX}.
    \end{itemize}
  
  \subsection{ Using the template locally }
    Coming soon.

  \subsection{ Preamble }
    Every \LaTeX document is divided in two sections, which we'll call \textit{preamble} and 
    \textit{content}.
    
    In the \textit{preamble} we'll define all the stuff that will be used in the rest of the 
    document (like the needed packages, commands, and configurations).

    Let's review the first few lines:
    \begin{minted}{tex}
      \documentclass{article}
      \usepackage{import} % This will make our life easier
      \import{preamble}{Packages}
      \import{preamble}{Definitions}
      \import{preamble}{config}
    \end{minted}

    The first line (\mintinline{tex}{\documentclass{article}}) defines the type of document we're
    writing, \texttt{article} is the most common, but it can also be \texttt{book}, \texttt{report},
    and many others (you can find a complete reference in \cite{CTANClass}).

    Next we have the \textit{imports}, if you're not interested in learning \LaTeX~and just want to
    use this template, you can ignore this part.
    I'll explain the specifics later on this document.

    Now let's get with the \texttt{setup}.
    The command is very flexible and should be used according to what you need, for example, you can
    set a subtitle in the \texttt{setup}, but if you don't plan to include a subtitle then you can 
    omit it.
    The syntax of the command is as follows:
    \begin{minted}{tex}
      \setup { 
        key1 = value1,
        key2 = {value 2}
      }
    \end{minted}
    where \texttt{keyn} is the name of a key (for example \texttt{title}), and \texttt{valuen} is the
    value of the key.
    Here, the braces \texttt{\{\}} are used to define a value that contains several words or a list
    of values.

    The available keys are:
    \begin{itemize}
      \item \texttt{title} - The title of the document.
      \item \texttt{subtitle} - The subtitle of the document.
      \item \texttt{author} - The author(s) of the document, you can specify multiple authors as a 
        list like \texttt{\{Author 1, Author 2\}}.
      \item \texttt{advisors} - This is meant to add other people involved in the document, such as 
        advisors, collaborators, reviewers, etc.
        The syntax is the same as for \texttt{author}.
      \item \texttt{logo} - The path to the logo of the document, this can be an absolute or 
        relative path (using a relative path is advised).
        The logo can be any valid image file or a pdf document.
      \item \texttt{location} - Where the document was written or published.
      \item \texttt{date} - The date of the document.
      \item \texttt{version} - The version of the document.
      \item \texttt{build} - The build number of the document (the \enquote{part 2} of the version).
      \item \texttt{commit} - The commit id of the document (the \enquote{part 3} of the version).
    \end{itemize}

    \textit{\textbf{Note:} The full version of the document will be defined as 
    \texttt{<version>.<build>.<commit>}.}

    As an example, this document was configured with the following setup:
    \begin{minted}{tex}
      \setup { 
        title = {
          A Heuristic Proof on the Non-Deterministic Behaviour of the \textit{Little Room}'s Printer
        },
        subtitle = { Thesis Proposal for the degree of Ph.D in Printer Sciences },
        author = {
          \textbf{DankMasterBlader} \\
          \textit{Institute of Printer Sciences} \\
          Letter Paper University of Applied Sciences
        },
        advisors = {
          \textbf{Alf} \\
          \textit{?¿??} \\
          University of \texttt{[BLANK]}. Ph.D  \\
          {\small\textit{Thesis Advisor}},
          \textbf{Juan Carlos Bodoque} \\
          \textit{Thirty-One Minutes} \\
          {\small\textit{Reviewer}}
        },
        logo = {logos/Logo_de_Plut_n.png},
        location = {Brooklyn, Japan},
        date = \today,
        % Please, I ask you for all that's precious in the world, always include a version for your 
        % document.
        version = 0,
        build = 1,
      }
    \end{minted}