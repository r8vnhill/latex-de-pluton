\section{ Quickstart }
  For the ones that want to just use this template and don't care about the details, this is the 
  section you'd want to focus on.

  \subsection{ Using the template with Overleaf }
    \begin{itemize}
      \item \textbf{Step 1:} \textit{Download the repository} as a zip file. 
        Alternatively, you can clone the repository and zip it manually.
      \item \textbf{Step 2:} \textit{In Overleaf, create a new project} and \textit{import} the zip 
        file.
        \texttt{New Project > Upload project}.
      \item \textbf{Step 3:} In the \textit{Project Menu} go to \texttt{Settings > Compiler} and 
        select \texttt{XeLaTeX}.
    \end{itemize}
  
  \subsection{ Using the template locally }
    Coming soon.

  \subsection{ Preamble }
    Every \LaTeX document is divided in two sections, which we'll call \textit{preamble} and 
    \textit{content}.
    
    In the \textit{preamble} we'll define all the stuff that will be used in the rest of the 
    document (like the needed packages, commands, and configurations).

    Let's review the first few lines:
    \begin{minted}{tex}
      \documentclass{article}
      \usepackage{import} % This will make our life easier
      \import{preamble}{Packages}
      \import{preamble}{Definitions}
      \import{preamble}{config}
    \end{minted}

    The first line (\mintinline{tex}{\documentclass{article}}) defines the type of document we're
    writing, \texttt{article} is the most common, but it can also be \texttt{book}, \texttt{report},
    and many others (you can find a complete reference in \cite{CTANClass}).

    Next we have the \textit{imports}, if you're not interested in learning \LaTeX~and just want to
    use this template, you can ignore this part.
    I'll explain the specifics later on this document.
